{\bfseries Proyecto de computación evolutiva}\par
\par
 Este proyecto implementa la simplificación de funciones booleanas, proceso que normalmente se utiliza con mapas de Karnaugh.\par
\par
 El proceso se utiliza con maxitérminos y minitérminos.\par
 \par
 La ejecución es de la siguiente forma\-:\par
\par
 {\ttfamily  ./exe -\/p $<$archivo de=\char`\"{}\char`\"{} entrada$>$=\char`\"{}\char`\"{}$>$ -\/o $<$archivo de=\char`\"{}\char`\"{} salida$>$=\char`\"{}\char`\"{}$>$ -\/i $<$n úmero=\char`\"{}\char`\"{} iteracciones=\char`\"{}\char`\"{} máxima$>$=\char`\"{}\char`\"{}$>$ -\/t $<$usar maxiterminos (0) o minitérminos$>$ -\/p $<$poblaci ón=\char`\"{}\char`\"{} inicial$>$=\char`\"{}\char`\"{}$>$ }\par
\par
 La entrada del problema debe ser el número de bits y una función binaria, con sus entradas ordenadas por ejemplo\-: {\ttfamily  2\par
 0 0 1\par
 0 1 0\par
 1 0 1\par
 1 1 0\par
 0 }\par
\par
 Pueden existir varias entradas, hasta que encuentra un 0 como número de bits\par
 \par
 La salida, es la ecuación del mejor cromosoma\par
 {\ttfamily  (x0 and x1) or (x0) } \begin{DoxyAuthor}{Autor}
Carlos Andres Delgado 

Edgar Andres Moncada 

Luis Felipe Vargas 
\end{DoxyAuthor}
\begin{DoxyVersion}{Versión}
1.\-0 
\end{DoxyVersion}
\begin{DoxyDate}{Fecha}
2012 
\end{DoxyDate}
\begin{DoxyRefDesc}{Bug}
\item[\hyperlink{bug__bug000001}{Bug}]No funciona con entradas con funciones desordenadas \end{DoxyRefDesc}
\begin{DoxyWarning}{Atención}
Evitar usar con entradas muy grandes 
\end{DoxyWarning}
\begin{DoxyCopyright}{Copyright}
G\-N\-U Public License. 
\end{DoxyCopyright}
